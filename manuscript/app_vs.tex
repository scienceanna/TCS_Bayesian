 %%%%%%%%%%%%%%%%%%%%%%% file template.tex %%%%%%%%%%%%%%%%%%%%%%%%%
%
% This is a general template file for the LaTeX package SVJour3
% for Springer journals.          Springer Heidelberg 2010/09/16
%
% Copy it to a new file with a new name and use it as the basis
% for your article. Delete % signs as needed.
%
% This template includes a few options for different layouts and
% content for various journals. Please consult a previous issue of
% your journal as needed.
%
%%%%%%%%%%%%%%%%%%%%%%%%%%%%%%%%%%%%%%%%%%%%%%%%%%%%%%%%%%%%%%%%%%%
%
% First comes an example EPS file -- just ignore it and
% proceed on the \documentclass line
% your LaTeX will extract the file if required
\begin{filecontents*}{example.eps}
%!PS-Adobe-3.0 EPSF-3.0
%%BoundingBox: 19 19 221 221
%%CreationDate: Mon Sep 29 1997
%%Creator: programmed by hand (JK)
%%EndComments
gsave
newpath
  20 20 moveto
  20 220 lineto
  220 220 lineto
  220 20 lineto
closepath
2 setlinewidth
gsave
  .4 setgray fill
grestore
stroke
grestore
\end{filecontents*}
%
\RequirePackage{fix-cm}
%
%\documentclass{svjour3}                     % onecolumn (standard format)
%\documentclass[smallcondensed]{svjour3}     % onecolumn (ditto)
\documentclass[smallextended]{svjour3}       % onecolumn (second format)
%\documentclass[twocolumn]{svjour3}          % twocolumn
%
\smartqed  % flush right qed marks, e.g. at end of proof
%
\usepackage{graphicx}
%
% \usepackage{mathptmx}      % use Times fonts if available on your TeX system
%
% insert here the call for the packages your document requires
%\usepackage{latexsym}
% etc.

%
% please place your own definitions here and don't use \def but
% \newcommand{}{}
%
% Insert the name of "your journal" with
% \journalname{myjournal}
%
\begin{document}

\title{Insert your title here%\thanks{Grants or other notes
%about the article that should go on the front page should be
%placed here. General acknowledgments should be placed at the end of the article.}
}
\subtitle{Do you have a subtitle?\\ If so, write it here}

%\titlerunning{Short form of title}        % if too long for running head

\author{Anna Hughes         \and
        Alasdair D. F. Clarke
}

%\authorrunning{Short form of author list} % if too long for running head

\institute{F. Author \at
              first address \\
              Tel.: +123-45-678910\\
              Fax: +123-45-678910\\
              \email{fauthor@example.com}           %  \\
%             \emph{Present address:} of F. Author  %  if needed
           \and
           S. Author \at
              second address
}

\date{Received: date / Accepted: date}
% The correct dates will be entered by the editor

\maketitle

\begin{abstract}
Insert your abstract here. Include keywords, PACS and mathematical
subject classification numbers as needed.
\keywords{First keyword \and Second keyword \and More}
% \PACS{PACS code1 \and PACS code2 \and more}
% \subclass{MSC code1 \and MSC code2 \and more}
\end{abstract}

\section{Introduction}
\label{intro}

Visual search, where participants are asked to find a target within a cluttered scene, has been extensively studied within psychology. A number of models have been developed that can generate testable predictions about how different types of distractors and targets affect search efficiency, including Feature Integration Theory \cite{treisman1980feature} and Guided Search \cite{wolfe1989guided,wolfe2014approaches}. These models aim to produce a representation of the visual properties of the distractors at each location in the visual field. However, a recent model, Target Contrast Signal (TCS) Theory, proposes that this is in fact not the goal of the visual system: the authors instead suggest that the most appropriate measure is how different each item is from the target \cite{lleras2020target}.

TCS proposes that behaviour is determined by comparing the target template (held in memory) with every element present in the scene in parallel. This allows the visual system to reject peripheral non-targets quickly; the speed at which items are evaluated is determined by how different the item is from the template through an evidence accumulation process. The model thus focuses on an initial, efficient processing stage of search; if sufficient evidence is not accumulated during this process, the model posits that a second stage is entered, where attention is deployed serially. TCS has been successful in predicting a number of empirical results, including search performance in heterogeneous scenes based on parameters estimated in homogeneous scenes, both with artificial stimuli \cite{buetti2016towards,lleras2019predicting} and with real-world objects visualised on a computer display \cite{wang2017predicting}. 

While results to date for TCS appear promising, they remain relatively preliminary, being tested by only one research group so far. In addition, predictions of search efficiency (e.g. in heterogeneous scenes) are made on the average of a group of participants, using data from a different group performing a different task (e.g. searching in homogeneous scenes). Thus, we know that TCS can replicate group-level averages in search well, but we do not know whether it is also able to make predictions at the individual level. There is also scope for extending beyond the parameters tested to date (such as the number of distractors) in order to test the robustness and generalisability of the model. Finally, there are some predictions, including its ability to explain search asymmetries, that have yet to be empirically confirmed \cite{lleras2020target}.

In the current manuscript, we focus on replicating and extending findings from \cite{buetti2019predicting}. Here, participants searched for a target in a scene of homogeneous distractors. First, parallel search efficiency (measured by the logarithmic search slope) was estimated for cases where the distractors varied from the target in one dimension: either colour (e.g. a cyan target being searched for in either yellow, blue or orange distractors) or shape (e.g. a semicircle target in either circle, diamond or triangle distractors). New participants then searched for the same targets in displays where the distractors were compounds, differing from the target in both colour and shape (e.g. searching for a cyan semicircle in either blue circles, orange diamonds or yellow triangles). The logarithmic search slopes in the initial experiments were then used to predict the logarithmic slopes and reaction times using a number of models. The authors found that the best model was a 'collinear contrast integration model' where the distinctiveness scores were summed along each attribute in the unidimensional experiments, creating an overall contrast score that was used for compound stimuli predictions.

\subsection{Hypotheses}

We plan a number of experiments to test the extent to which the original results in \cite{buetti2019predicting} replicate and generalise. As well as following the original, between-subjects, experiment design, in which data from one group of observers in one task is used to predict behaviour of a second group of participants in a different task, we will allow for within-subject comparisons. Specifically, we will ask: to what extent do the individual differences in the homogeneous task explain the differences in the heterogeneous task? 

\begin{enumerate}
\item \textbf{Replication of Buetti et al (2019) with online data collection.} Specifically, that the \textit{collinear contrast ingratiation model} outperforms the \textit{best feature guidance}, and \textit{orthogonal contrast combination models}.  Furthermore, the $R^2 = $ ($99\%$ HPDI = $[, ]$) between predicted and observer reaction times.\\
\item Larger number of distractors (or targets further in periphery?) Add an extra ring \\ 
\item Search asymmetries: test O in Qs and Q in Os \\
\end{enumerate}

\section{Reanalysis of Buetti et al (2019)}

In our proposed experiments, we would like to:
\begin{itemize} 
\item make use of multi-level models \\
\item, and work within a Bayesian framework \\
\item use a log-normal distribution to take skew in rt distribution into account \\
\end{itemize}

To start, we will re-analysis previous data to verify that this does not invalidate the original conclusions. And, how to phrase, something about these new results (on the old data) being the ones we want to replicate. 

\subsection{Methods}

\subsubsection{Data}

Data taken from OSF... using the exclusion criteria originally used. What about incorrect trials etc? 

Do we want to re-do any of the other papers while we're at it?

There is an rt = 0, which I will remove. 

\subsubsection{Modelling Approach}

R, brms, details.

We will compute the $D$ values using a multi-level Bayesian model using a log-normal distribution. In order to keep our model in line with the Target Contrast Signal Theory, we will use the identity link rather than the default log link. We will also model reaction times on the seconds scale, rather than milliseconds. While this is a fairly trivial change, it makes numerical model fitting slightly easier. 

\begin{equation}
\hat{y} = a + Dlog(N_T + 1)
\label{eq:computeDlm}
\end{equation}

For now, I'm just using random intercept. Hope to implement full random slopes soon!



Funky equation from Buetti. 

\subsubsection{Prior Predictions}

\subsection{Results}

\subsubsection{Fixed Effects}

As we are using a Bayesian framework, rather than $D$ representing the best fit in equation \ref{eq:computeDlm}, it is now represented as a probability distribution over possible values (see Figure \ref{fig:reanalysisBuetti2019}.)


\begin{figure}
\caption{Posterior predictions of $D$ for different distracter features. The vertical lines indicate the (transformed) values given in Buetti (2019).}
\label{fig:reanalysisBuetti2019}
\end{figure} 

\subsubsection{Individual Differences}

\subsection{Discussion}


\section{Experiment 1}

H0: confirm collinear method is best (compared to orthogonal contrast combination and best feature guidance models)

[If this is indeed the case, we will use only the collinear contrast integration model in the following experiments].

\subsection{Methods}

\subsection{Results}

\subsection{Discussion}



\begin{acknowledgements}
Thank you to AL for help and encouragement! 
\end{acknowledgements}


% Authors must disclose all relationships or interests that 
% could have direct or potential influence or impart bias on 
% the work: 
%
% \section*{Conflict of interest}
%
% The authors declare that they have no conflict of interest.


% BibTeX users please use one of
\bibliographystyle{plain}      % basic style, author-year citations
%\bibliographystyle{spmpsci}      % mathematics and physical sciences
%\bibliographystyle{spphys}       % APS-like style for physics
\bibliography{sources}   % name your BibTeX data base

% Non-BibTeX users please use
%\begin{thebibliography}{}
%
% and use \bibitem to create references. Consult the Instructions
% for authors for reference list style.
%
%\bibitem{RefJ}
% Format for Journal Reference
%Author, Article title, Journal, Volume, page numbers (year)
% Format for books
%\bibitem{RefB}
%Author, Book title, page numbers. Publisher, place (year)
% etc
%\end{thebibliography}

\end{document}
% end of file template.tex

