 %%%%%%%%%%%%%%%%%%%%%%% file template.tex %%%%%%%%%%%%%%%%%%%%%%%%%
%
% This is a general template file for the LaTeX package SVJour3
% for Springer journals.          Springer Heidelberg 2010/09/16
%
% Copy it to a new file with a new name and use it as the basis
% for your article. Delete % signs as needed.
%
% This template includes a few options for different layouts and
% content for various journals. Please consult a previous issue of
% your journal as needed.
%
%%%%%%%%%%%%%%%%%%%%%%%%%%%%%%%%%%%%%%%%%%%%%%%%%%%%%%%%%%%%%%%%%%%
%
% First comes an example EPS file -- just ignore it and
% proceed on the \documentclass line
% your LaTeX will extract the file if required
\begin{filecontents*}{example.eps}
%!PS-Adobe-3.0 EPSF-3.0
%%BoundingBox: 19 19 221 221
%%CreationDate: Mon Sep 29 1997
%%Creator: programmed by hand (JK)
%%EndComments
gsave
newpath
  20 20 moveto
  20 220 lineto
  220 220 lineto
  220 20 lineto
closepath
2 setlinewidth
gsave
  .4 setgray fill
grestore
stroke
grestore
\end{filecontents*}
%
\RequirePackage{fix-cm}
%
%\documentclass{svjour3}                     % onecolumn (standard format)
%\documentclass[smallcondensed]{svjour3}     % onecolumn (ditto)
\documentclass[smallextended]{svjour3}       % onecolumn (second format)
%\documentclass[twocolumn]{svjour3}          % twocolumn
%
\smartqed  % flush right qed marks, e.g. at end of proof
%
\usepackage{graphicx}
%
% \usepackage{mathptmx}      % use Times fonts if available on your TeX system
%
% insert here the call for the packages your document requires
%\usepackage{latexsym}
% etc.
%
% please place your own definitions here and don't use \def but
% \newcommand{}{}
%
% Insert the name of "your journal" with
% \journalname{myjournal}
%
\begin{document}

\title{Insert your title here%\thanks{Grants or other notes
%about the article that should go on the front page should be
%placed here. General acknowledgments should be placed at the end of the article.}
}
\subtitle{Do you have a subtitle?\\ If so, write it here}

%\titlerunning{Short form of title}        % if too long for running head

\author{Anna Hughes         \and
        Alasdair D. F. Clarke
}

%\authorrunning{Short form of author list} % if too long for running head

\institute{F. Author \at
              first address \\
              Tel.: +123-45-678910\\
              Fax: +123-45-678910\\
              \email{fauthor@example.com}           %  \\
%             \emph{Present address:} of F. Author  %  if needed
           \and
           S. Author \at
              second address
}

\date{Received: date / Accepted: date}
% The correct dates will be entered by the editor

\maketitle

\begin{abstract}
Insert your abstract here. Include keywords, PACS and mathematical
subject classification numbers as needed.
\keywords{First keyword \and Second keyword \and More}
% \PACS{PACS code1 \and PACS code2 \and more}
% \subclass{MSC code1 \and MSC code2 \and more}
\end{abstract}

\section{Introduction}
\label{intro}

Target Contrast Signal Theory: "precise mathematical model that allows one to make specific point predictions about how components of visual complexity combine to impact human performance".

Model predicts search performance in heterogeneous scenes based on parameters estimated in homogeneous scenes. 


\paragraph{}

``Wang, Buetti and Lleras (2017) developed an equation to predict search performance in heterogeneous visual search scenes (i.e., multiple types of non-target objects simultaneously present) based on parameters observed when participants perform search in homogeneous scenes (i.e., when all non-target objects are identical to one another). The equation was based on a computational model where every item in the display is processed with unlimited capacity and independently of one another, with the goal of determining whether the item is likely to be a target or not. The model was tested in two experiments using real-world objects.''

\paragraph{}

\textbf{Lleras (2019):}

Expt 1: heterogeneous search using Buetti et al's (2016) stimuli.
Target - red triangle.
Distractors - orange diamonds, yellow triangles, blue circles.
Search items randomly assigned positions on invisible 6x6 rectangular grid with jitter. 2 types of experimental conditions - 2 mixed or 3 mixed. For each group, arbitrarily held constant number of items of one lure type, and then manipulated number of items of other 2 lure types.
Search performance in heterogeneous search displays is very well predicted by performance observed in homogeneous search displays (from Expt 1A, Buetti et al 2016).

Expt2A/2B: are heterogeneous spatially segregated displays processed with the same efficiency as entirely homogeneous displays? 2A - abstract shapes. 2B - real world stimuli. Also able to predict variability in performance quite well. Spatially segregated scenes producing search performance as efficient as completely homogeneous scenes.
Conclude that processing required to complete parallel search task in lure heterogeneous scenes is fundamentally same as in lure homogeneous scenes. Lure to lure stimuli interactions facilitate processing in parallel scenes and strength of those interactions maximal when identical lures near each other.


\textbf{Buetti (2019):} 

Parallel search efficiency (logarithmic search slope) to find target amongst homogeneous distractors estimated: different colours (red target in orange, blue, yellow distractors) or shapes (semicircle target in circle, diamond, triangle distractors) tested. 

New group of participants searched for same target in display where the distractors were compounds i.e. differed from target in both colour and shape (e.g. blue circles, orange diamonds, yellow triangles). Observed RTs in latter experiment compared to predicted reaction times from model. 

\textbf{Lleras (2020):}

Logarithmic dependency of RT on set size - indicative of processing architecture with parallel processing, unlimited capacity and exhaustive termination rule. Can account for modulation of steepness of logarithmic functions by assuming processing time of individual item is inversely proportional to its dissimilarity to target.

Lures are distractors that are sufficiently different from the target (items that have reached threshold). [Ed note - what is sufficiently in this context?] Candidates are distractors that are similar to the target (items that have yet to reach threshold) - processed in second, capacity limited stage. 

What TCS can do:

\begin{enumerate}
\item TCS provides account for variation of RT as function of set size as observed in efficient search tasks. Logarithmic search slope coefficient measured in typical efficient search task (one target, all identical lures) indexes dissimilarity between target and lure stimuli \\
\item Logarithmic slope coefficients can be used to predict performance in future experiments where same target and lure stimuli are paired \\
\item Provides framework for understanding occurrence of eye movements to non target elements \\
\item Framework for evaluating and quantifying lure-lure similarity effects in efficient search \\
\item Framework for understanding processing costs associated with varying levels of target-distractor similarity \\
\item Intuitive account for search asymmetries: target-contrast signal computed as function of visual properties present in target template \\
\item Framework for understanding impact of crowding, stimulus size - integral to determining quality of peripheral representations \\
\end{enumerate}

Evidence accumulation process consists of an evaluation of an evaluation of the extent to which properties at each location differ from the set of properties that define the target. Properties of the distractor that are not present in the target are ignored.

\textbf{Ng et al, submitted):}

Lack of prioritisation: displays contained two different types of candidates - some extremely similar, some less so. Failed to find any evidence that more similar candidates were prioritised over less similar candidates.

\textbf{Buetti (2016):}
Expt 2 shows that inefficient search is linear and self terminating using target T in Ls. 
Expt 3 claims to show support for distinction between candidates and lures.

\subsection{Hypothesis}

We plan a number of experiments to test the extent to which the original results replicate and generalise. As well as following the original, between-subjects, experiment design, in which data from one group of observers in one task is used to predict behaviour of a second group of participants in a different task, we will allow for within-subject comparisons. Specifically, to what extent do the individual differences in the homogeneous task explain the differences in the heterogeneous task? 

\begin{enumerate}
\item \textbf{Replication of Buetti et al (2019) with online data collection.} Specifically, that the \textit{collinear contrast ingratiation model} outperforms the \textit{best feature guidance}, and \textit{orthogonal contrast combination models}.  Furthermore, the $R^2 = $ ($99\%$ HPDI = $[, ]$) between predicted and observer reaction times.\\
\item Larger number of distractors (or targets further in periphery?) Add an extra ring \\ 
\item Search asymmetries: test O in Qs and Q in Os \\
\end{enumerate}

\section{Reanalysis of Buetti et al (2019)}

In our proposed experiments, we would like to:
\begin{itemize} 
\item make use of multi-level models, \\
\item, and work within a Bayesian framework, \\
\item use a log-normal distribution to take skew in rt distribution into account \\
\end{itemize}

To start, we will re-analysis previous data to verify that this does not invalidate the original conclusions. And, how to phrase, something about these new results (on the old data) being the ones we want to replicate. 

\subsection{Methods}

\subsubsection{Data}

Data taken from OSF... using the exclusion criteria originally used. What about incorrect trials etc? 

Do we want to re-do any of the other papers while we're at it?

There is an rt = 0, which I will remove. 

\subsubsection{Modelling Approach}

R, brms, details.

We will compute the $D$ values using a multi-level Bayesian model using a log-normal distribution. In order to keep our model in line with the Target Contrast Signal Theory, we will use the identity link rather than the default log link. We will also model reaction times on the seconds scale, rather than milliseconds. While this is a fairly trivial change, it makes numerical model fitting slightly easier. 

\begin{equation}
\hat{y} = a + Dlog(N_T + 1)
\label{eq:computeDlm}
\end{equation}

For now, I'm just using random intercept. Hope to implement full random slopes soon!



Funky equation from Buetti. 

\subsubsection{Prior Predictions}

\subsection{Results}

\subsubsection{Fixed Effects}

As we are using a Bayesian framework, rather than $D$ representing the best fit in equation \ref{eq:computeDlm}, it is now represented as a probability distribution over possible values (see Figure \ref{fig:reanalysisBuetti2019}.)


\begin{figure}
\caption{Posterior predictions of $D$ for different distracter features. The vertical lines indicate the (transformed) values given in Buetti (2019).}
\label{fig:reanalysisBuetti2019}
\end{figure} 

\subsubsection{Individual Differences}

\subsection{Discussion}


\section{Experiment 1}

H0: confirm collinear method is best (compared to orthogonal contrast combination and best feature guidance models)

[If this is indeed the case, we will use only the collinear contrast integration model in the following experiments].

\subsection{Methods}

\subsection{Results}

\subsection{Discussion}



\begin{acknowledgements}
Thank you to AL for help and encouragement! 
\end{acknowledgements}


% Authors must disclose all relationships or interests that 
% could have direct or potential influence or impart bias on 
% the work: 
%
% \section*{Conflict of interest}
%
% The authors declare that they have no conflict of interest.


% BibTeX users please use one of
%\bibliographystyle{spbasic}      % basic style, author-year citations
%\bibliographystyle{spmpsci}      % mathematics and physical sciences
%\bibliographystyle{spphys}       % APS-like style for physics
%\bibliography{}   % name your BibTeX data base

% Non-BibTeX users please use
\begin{thebibliography}{}
%
% and use \bibitem to create references. Consult the Instructions
% for authors for reference list style.
%
\bibitem{RefJ}
% Format for Journal Reference
Author, Article title, Journal, Volume, page numbers (year)
% Format for books
\bibitem{RefB}
Author, Book title, page numbers. Publisher, place (year)
% etc
\end{thebibliography}

\end{document}
% end of file template.tex

