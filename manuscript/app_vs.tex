 %%%%%%%%%%%%%%%%%%%%%%% file template.tex %%%%%%%%%%%%%%%%%%%%%%%%%
%
% This is a general template file for the LaTeX package SVJour3
% for Springer journals.          Springer Heidelberg 2010/09/16
%
% Copy it to a new file with a new name and use it as the basis
% for your article. Delete % signs as needed.
%
% This template includes a few options for different layouts and
% content for various journals. Please consult a previous issue of
% your journal as needed.
%
%%%%%%%%%%%%%%%%%%%%%%%%%%%%%%%%%%%%%%%%%%%%%%%%%%%%%%%%%%%%%%%%%%%
%
% First comes an example EPS file -- just ignore it and
% proceed on the \documentclass line
% your LaTeX will extract the file if required
\begin{filecontents*}{example.eps}
%!PS-Adobe-3.0 EPSF-3.0
%%BoundingBox: 19 19 221 221
%%CreationDate: Mon Sep 29 1997
%%Creator: programmed by hand (JK)
%%EndComments
gsave
newpath
  20 20 moveto
  20 220 lineto
  220 220 lineto
  220 20 lineto
closepath
2 setlinewidth
gsave
  .4 setgray fill
grestore
stroke
grestore
\end{filecontents*}
%
\RequirePackage{fix-cm}
%
%\documentclass{svjour3}                     % onecolumn (standard format)
%\documentclass[smallcondensed]{svjour3}     % onecolumn (ditto)
\documentclass[smallextended]{svjour3}       % onecolumn (second format)
%\documentclass[twocolumn]{svjour3}          % twocolumn
%
\smartqed  % flush right qed marks, e.g. at end of proof
%
\usepackage{graphicx}
%
% \usepackage{mathptmx}      % use Times fonts if available on your TeX system
%
% insert here the call for the packages your document requires
%\usepackage{latexsym}
% etc.
%
% please place your own definitions here and don't use \def but
% \newcommand{}{}
%
% Insert the name of "your journal" with
% \journalname{myjournal}
%
\begin{document}

\title{Insert your title here%\thanks{Grants or other notes
%about the article that should go on the front page should be
%placed here. General acknowledgments should be placed at the end of the article.}
}
\subtitle{Do you have a subtitle?\\ If so, write it here}

%\titlerunning{Short form of title}        % if too long for running head

\author{First Author         \and
        Second Author %etc.
}

%\authorrunning{Short form of author list} % if too long for running head

\institute{F. Author \at
              first address \\
              Tel.: +123-45-678910\\
              Fax: +123-45-678910\\
              \email{fauthor@example.com}           %  \\
%             \emph{Present address:} of F. Author  %  if needed
           \and
           S. Author \at
              second address
}

\date{Received: date / Accepted: date}
% The correct dates will be entered by the editor

\maketitle

\begin{abstract}
Insert your abstract here. Include keywords, PACS and mathematical
subject classification numbers as needed.
\keywords{First keyword \and Second keyword \and More}
% \PACS{PACS code1 \and PACS code2 \and more}
% \subclass{MSC code1 \and MSC code2 \and more}
\end{abstract}

\section{Introduction}
\label{intro}

Target Contrast Signal Theory: "precise mathematical model that allows one to make specific point predictions about how components of visual complexity combine to impact human performance".

Model predicts search performance in heterogeneous scenes based on parameters estimated in homogeneous scenes. 

\paragraph{}

Buetti (2019): Parallel search efficiency (logarithmic search slope) to find target amongst homogeneous distractors estimated: different colours (red target in orange, blue, yellow distractors) or shapes (semicircle target in circle, diamond, triangle distractors) tested. 
New group of participants searched for same target in heterogeneous displays that contained multiple types of distractors (e.g. blue circles, orange diamonds, yellow triangles). Observed RTs in latter experiment compared to predicted reaction times from model.

\paragraph{} 

\subsection{Hypothesis}

We plan a number of experiments to test the extent to which the original results replicate and generalise. As well as following the original, between-subjects, expeirment design, in which data from one group of observers in one task is used to predict behaiourt of a second group of participatns in a different task, we will allow for within-subject comparisons. Specifically, to what extent do the indvidual differences in the homogeneous task explain the differences in the heteregeneous task? 

\begin{enumerate}
\item \textbf{Replication of Buetti et al (2019) with online data collection.} Specificially, that the \textit{collinear contrast ingeration model} outperforms the \textit{best feature guidance}, and \textit{orthogoncal contrast combination models}.  Furthermore, the $R^2 = $ ($99\%$ HPDI = $[, ]$) between predicted and observer reaction times.\\
\item non-independent features \\
\item Larger number of distractors \\ 
\item Larger number of distractor types \\ 
\end{enumerate}

\section{Reanalysis of Buetti et al (2019)}

In our proposed experiments, we would like to make use of multi-level models, and work within a Bayesian framework. To start, we will re-analysis previous data to verfiy that this does not invalidate the original conclusions. And, how to phrase, something about these new results (on the old data) being the ones we want to replicate. 

\subsection{Methods}

\subsubsection{Data}

Data taken from OSF... using the exclsion criteria originally used. What about incorrect trials etc? 

Do we want to re-do any of the other papers while we're att it?

\subsubsection{Modelling Approach}

R, brms, details.

Funky equation from Buetti. 

\subsubsection{Prior Predictions}

\subsection{Results}

\subsection{Discussion}


\section{Experiment 1}

H0: confirm collinear method is best (compared to orthogonal contrast combination and best feature guidance models)

[If this is indeed the case, we will use only the collinear contrast integration model in the following experiments].

\subsection{Methods}

\subsection{Results}

\subsection{Discussion}



\begin{acknowledgements}
Thank you to AL for help and encouragement! 
\end{acknowledgements}


% Authors must disclose all relationships or interests that 
% could have direct or potential influence or impart bias on 
% the work: 
%
% \section*{Conflict of interest}
%
% The authors declare that they have no conflict of interest.


% BibTeX users please use one of
%\bibliographystyle{spbasic}      % basic style, author-year citations
%\bibliographystyle{spmpsci}      % mathematics and physical sciences
%\bibliographystyle{spphys}       % APS-like style for physics
%\bibliography{}   % name your BibTeX data base

% Non-BibTeX users please use
\begin{thebibliography}{}
%
% and use \bibitem to create references. Consult the Instructions
% for authors for reference list style.
%
\bibitem{RefJ}
% Format for Journal Reference
Author, Article title, Journal, Volume, page numbers (year)
% Format for books
\bibitem{RefB}
Author, Book title, page numbers. Publisher, place (year)
% etc
\end{thebibliography}

\end{document}
% end of file template.tex

